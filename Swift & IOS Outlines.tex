\documentclass[14pt]{extarticle} 
\usepackage[top=2cm,bottom=2cm,left=3cm,right=2cm]{geometry}
%\usepackage{listings}
%\lstinputlisting[language=python]{mypythonfile.py}
\usepackage{graphicx}
\usepackage{listings}
\usepackage[normalem]{ulem}
\usepackage[]{natbib}
\usepackage{titlepic}
\usepackage{fancyhdr}
\usepackage{pdfpages}


\pagestyle{headings}
\title{Swift \& IOS Outlines}
\author{Getting Started}
\date{}
\titlepic{\includegraphics[width=0.7\linewidth]{C:/Users/HP/Downloads/logo}} 

\begin{document}
	\includepdf[pages=1,fitpaper]{C:/Users/HP/Downloads/swift}
	\setlength{\baselineskip}{25 pt}


    \maketitle 
    \newpage
	
	\tableofcontents
	\newpage
	% Here is the part of the .tex file that you have to worry about!
\section{Introduction}
Swift Programming Language

\begin{itemize}
	\item Data Types, Constants, and Variables
	\item Operators and Expressions
	\item Object-oriented programming in Swift
	\item An Introduction to Swift Sub-classing and Extensions
	\item Working with Array and Dictionary Collections in Swift
	\item Understanding Error Handling in Swift
	\item Protocols
	\item Functions
\end{itemize}

like:
\begin{lstlisting}[language=python]
	
	func greet(person: String) -> String { 
		
 	  let greeting = "Hello, " + person + "!"
	  return greeting
	
	}
	
\end{lstlisting}

	
\section{Views, Layouts,and Storyboards}
\begin{itemize}
	\item An Introduction to Auto Layout in iOS Implementing iOS Auto LayoutConstraints in Code
	\item Understanding the iOS Auto Layout
	\item VisualFormat Language
	\item Using Storyboards in Xcode
	\item Using Xcode Storyboards to Create an iOS Tab Bar Application
	\item Working with the iOS Stack View Class
	\item Implementing a Page based iOS Application using UIPageViewController
	\item Navigation Controllers
	\item Table Views
	\item Collection Views
	\item Tab Bars and Toolbars
	\item Creating Views That Scroll
	\item Popovers and Modal Views
\end{itemize}

\section{Data Storagewith Files and Databases}
\begin{itemize}
	\item Working with Directories
	\item Working with Files
	\item Managing Files using the iOS UIDocument Class
	\item Working with iOS Databases using Core Data
\end{itemize}

\section{Touch, Taps, and Gestures}
\begin{itemize}
	\item Multitouch, Taps and Gestures
	\item Detecting iOS Touch ScreenGesture Motions
	\item Identifying Gesturesusing iOS	GestureRecognizers
\end{itemize}

\section{Consuming RESTfulJSON Web Services}
\begin{itemize}
	\item Types of Web Services
	\item Pods \& AlamoFire
\end{itemize}

\section{Extensions}
\begin{itemize}
	\item An Extension Widget
	\item Creating an iOS Photo EditingExtension
	\item Creating an iOS Action Extension
	\item Receiving Data from an iOS Action Extension
\end{itemize}

\section{Multimedia}
\begin{itemize}
	\item Accessing the iOS Camera and Photo Library
	\item iOS Video Playback using AVPlayer and
	AVPlayerViewController
	\item Playing Audio on iOS using AVAudioPlayer
	\item Recording Audio on iOS with AVAudioRecorder
\end{itemize}

\section{Location}
\begin{itemize}
	\item Permissions \& Accuracy
	\item Receiving LocationUpdates
	\item Geocoding and Reverse Geocoding
	\item Obtaining CompassHeadings
\end{itemize}

\section{Map Kit}
\begin{itemize}
	\item Adding Annotations
	\item Accessory Views
	\item Routing with MapKit and Core Location
\end{itemize}

\section{Design Patterns}
\begin{itemize}
	\item Singleton
	\item MVVM \& MVC \& MVP \& VIPER
\end{itemize}

\section{Data management}
\begin{itemize}
	\item Importance of data management
	\item Types of management functions
	\item Tools and techniques
	\item tasks and roles
\end{itemize}

\section{Unit testing \& UI testing}
\begin{itemize}
	\item Creating a Unit Test Target
	\item Using XCTAssert to Test Models
	\item Writing Your First Test
	\item Debugging a Tes
	\item Failing Conditionally
	\item Faking Objects and Interactions
	\item Faking Input From Stub
	\item Testing Performance
	\item Enabling Code Coverage
\end{itemize}

\section{Different topics}
\begin{itemize}
	\item ARKit introduction
	\item Dependency injection
	\item Objective-C generalknowledge
\end{itemize}

\section{The App Store}
\begin{itemize}
	\item Creating an Application Profile
	\item Preparing and Uploading the Application Binary
	\item Submitting the App for Review
\end{itemize}


\section{Functions}



\subsection{Defining Functions}
\begin{lstlisting}[language=python]
func functionName(parameters) -> returnType { 
	//functionBody 
  }

func greet(person: String) -> String { 
	let greeting = "Hello, " + person + "!"
	return greeting
  }
	
\end{lstlisting}



\subsection{Calling Functions}
\begin{lstlisting}[language=python]
 print(greet(person: "Anna")) 
   // Prints "Hello, Anna!"
 print(greet(person: "Brian")) 
   // Prints "Hello, Brian!"
	
\end{lstlisting}

\subsection{Functions Without Parameters}
\begin{lstlisting}[language=python]
  func sayHello() -> String { 
	return "hello, world"
  } 
	
  print(sayHello()) 
    // Prints "hello, world"
\end{lstlisting}


\subsection{Functions With Multiple Parameters}
\begin{lstlisting}[language=python]
func greet(person: String, alreadyGreeted: Bool) -> String { 
	if alreadyGreeted { 
		return greetAgain(person: person) 
	} else { 
		return greet(person: person) 
  }  
} 	
print(greet(person: "Tim", alreadyGreeted: true)) 
// Prints "Hello again, Tim!"
\end{lstlisting}


\subsection{Functions Without Return Values}
\begin{lstlisting}[language=python]
func greet(person: String) { 
	print("Hello, \(person)!") 	
} 
greet(person: "Dave") 
  // Prints "Hello, Dave!"
\end{lstlisting}


\subsection{Functions with Multiple Return Values}
\begin{lstlisting}[language=python]
func minMax(array: [Int]) -> (min: Int, max: Int) { 
	var currentMin = array[0] 
	var currentMax = array[0] 
	for value in array[1..<array.count] { 
		if value < currentMin { 
			currentMin = value
		} else if value > currentMax { 
			currentMax = value
	} 	
  } 
		
	return (currentMin, currentMax) 
}
\end{lstlisting}


\subsection{Functions With an Implicit Return}
\begin{lstlisting}[language=python]
func greeting(for person: String) -> String { 
	"Hello, " + person + "!"
}
print(greeting(for: "Dave")) 
  // Prints "Hello, Dave!"
\end{lstlisting}

\subsection{Function Argument Labels and Parameter Names}
\begin{lstlisting}[language=python]
func sum(x: Int, y: Int) { 
	// In the function body, x and y
	// refer to the argument values
} 
sum(x: 1, y: 2)
\end{lstlisting}

\subsection{Specifying Argument Labels}
\begin{lstlisting}[language=python]
func greet(person: String, from hometown: String) -> String { 
	return "Hello \(person)! Glad you could visit from \(hometown)."	
}
print(greet(person: "Bill", from: "Cupertino"))
\end{lstlisting}

\subsection{Variadic Parameters}
\begin{lstlisting}[language=python]
func arithmeticMean(_ numbers: Double...) -> Double { 
	var total: Double = 0
	for number in numbers { 
		total += number
	} 
	return total / Double(numbers.count) 
}
arithmeticMean(1, 2, 3, 4, 5) 
  // returns 3.0, which is the arithmetic mean of these five numbers
arithmeticMean(3, 8.25, 18.75) 
  // returns 10.0, which is the arithmetic mean of these three numbers
\end{lstlisting}

\subsection{In-Out Parameters}
\begin{lstlisting}[language=python]
func swapTwoInts(_ x: inout Int, _ y: inout Int) { 
	let temporary = x
	x = y
	y = temporary
}
	var a = 10
	var b = 20
	swapTwoInts(&a, &b) 
print("a is now \(a), and b is now \(b)") 
\end{lstlisting}



\begin{figure}
	\section{Install Xcode}	
	\centering
	\includegraphics[width=1\linewidth]{C:/Users/HP/OneDrive/Desktop/R}
	\caption{Once you have Xcode installed, you should see the following welcome dialog:}
	\label{fig:r}
\end{figure}

\begin{figure}
	\centering
	\includegraphics[width=0.8\linewidth]{C:/Users/HP/OneDrive/Desktop/name}
	\caption{Choose a project name}
	\label{fig:name}
\end{figure}
\begin{figure}
	\section{Creating an Xcode Project}
	\centering
	\includegraphics[width=0.8\linewidth]{C:/Users/HP/OneDrive/Desktop/ss}
	\caption{Different Elements of the Playground}
	\label{fig:ss}
\end{figure}

\begin{figure}
	\section{Different Elements of the Playground}
	\centering
	\includegraphics[width=0.8\linewidth]{C:/Users/HP/OneDrive/Desktop/kit}
	\caption{Code editor : this is where you’re going to be typing your Swift code.}
	\label{fig:kit}
\end{figure}
\begin{figure}
	\centering
	\includegraphics[width=0.8\linewidth]{C:/Users/HP/OneDrive/Desktop/read}
	\caption{Status bar : tells you the current status of the playground.}
	\label{fig:read}
\end{figure}

\begin{figure}
	\centering
	\includegraphics[width=0.8\linewidth]{C:/Users/HP/OneDrive/Desktop/d}
	\caption{Show/Hide Debug : allows you to hide or show the debug or console.}
	\label{fig:d}
\end{figure}

\begin{figure}
	\centering
	\includegraphics[width=0.8\linewidth]{C:/Users/HP/OneDrive/Desktop/run}
	\caption{Execute Playground : runs all the code in your playground.}
	\label{fig:run}
\end{figure}

\begin{figure}
	\section{It's Ready}
	\centering
	\includegraphics[width=0.8\linewidth]{C:/Users/HP/OneDrive/Desktop/code}
	\caption{xcode Ready to run your own project, Enjoy coding :)}
	\label{fig:code}
\end{figure}



\end{document}	